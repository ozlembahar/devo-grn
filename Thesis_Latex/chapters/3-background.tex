\chapter{Background}\label{chap:background}
\section{Central Nervous System Development in \textit{Drosophila} melanogaster}
The central nervous system (CNS) of the \textit{Drosophila} embryo is a complex organ encompassing approximately 15,000 
neurons and glial cells and develops over the course of roughly one day \cite{Crews2019,ito1995distribution,heckscher2014atlas,cobeta2017anterior,yaghmaeian2018evolutionarily}. 
This system is composed of the brain and the ventral nerve cord (VNC) and is structured into segments known as neuromeres. 
In the VNC, Each neuromere is subdivided by a set of midline cells into two bilaterally symmetric hemineuromeres, 
or hemisegments \cite{urbach2016gene,Crews2019}. The VNC itself is comprised of a total of 28 hemineuromeres. 
%% Oz: this includes thorax and brain segments, we are interested in abdominal segments (A1-A12) --> I would just skip this info then
The VNC is generated in the ventral-lateral region of the embryo called the neuroectoderm \cite{Crews2019}.  

\subsubsection{Neuroblasts Formation and Specification in the VNC During Neurogenesis}
Neuroblasts are the progenitor cells that give rise to the diverse neuronal and glial populations within the Drosophila CNS. 
During neurogenesis NBs within the hemisegments of the VNC are emerging by delamination from a sheet of surrounding cells in 
five consecutive waves (S1 to S5) \cite{bossing1996embryonic}, and are arranged into a pattern of rows and columns. 
The formation and fate specification of neuroblasts are uniform across all hemisegments. Different NBs give rise to different 
types of neurons or glia cells later in development. These NBs' fates are determined by their relative position within the 
hemisegment and time of formation. 
That means that NBs in different hemisegments, that are developed in the same row and column within the hemisegment, 
acquire the same fate. 
Eventually each NB undergoes division to generate a distinct group of neurons or glial cells that make up a functioning CNS 
\cite{Skeath1999}. 
%  Oz: In general it is hard to follow whats coming from where, what generates what…
% My suggesting would be to summarize generally what NBs are, where they are located and how they are generated in a 
% structured way so that it becomes easier to follow for someone how does not know anything about this subject. 

\subsubsection{Neuroectoderm Patterning}
Neuroblasts formation in the neuroectoderm takes place between stages 8-9 to late stage 11  of development, 
during approximately four hours. Each hemineuromer counts about 10 NBs at the beginning of the NB formation process, 
and ends up with about 32 \cite{Crews2019, Bhat1999, bossing1996embryonic, schmidt1997embryonic}.  
%At the end of neurogenesis, each of the hemineuromeres has roughly 320 neurons and 30 glia cells 

NBs are formed within specialized groups of cells known as preneural clusters, 
where the specific pattern of gene expression within these clusters dictates the identity and eventual fate of each NB. 
In each hemisegment NBs are organized initially into a grid of four rows and three columns oriented along 
the anteroposterior (AP) and dorsoventral (DV) axes of the embryo, respectively \cite{Crews2019, Bhat1999, Skeath1999}.  
%Eventually each hemi-segment, there exist around seven rows of NBs.

The patterning of the neuroectoderm is regulated by the interplay between a number of signaling pathways. 
% Oz:  You talk about activitiy of different pathways before introducing the pathways—Too much info already
The activity of different pathways in different rows and columns of the hemisegment result in different gene expression 
profile of each NB depending on it's position, hence determining its identity and unique fate. 
Segmentation genes were found to be involved in NBs specification as well, some of them as part of known signalling pathways. 
% Oz: Not introduced before- only said there are some genes expressed in specific patterns..
However the full regulatory mechanism in which signalling pathways employ different genes is not yet fully understood \cite{Bhat1999}. 
This remains a pursuit in developmental biology.   %It is still a pursuit in developmental biology. 
%Several segmentation genes were found to segment the neuroectoderm and take part in the formation and specification of NBs of the VNC. 
%The exact regulatory mechanism in charge of NBs specification during CNS formation remains not fully understood. 

\subsubsection{Signaling Pathways}
Several signaling pathways were found to play an important role in NBs formation and specification, 
such as Notch, Wingless (Wg/Wnt), Hedgehog (Hh) and Epidermal growth factor receptor (EGFr) signaling pathways. 

Notch signaling pathway was found to affect the formation of all NBs \cite{Crews2019}. 

Wg, Hh and EGFr pathways were shown to be involved in the separation of NBs to different rows and columns. 
More specifically, the interaction between Wg and Hh were found to pattern the embryo to different rows
along the AP axis in a repetitive manner. % Oz: what is repetitive exactly?  
For example, Wg pathway is active  in rows 4 and 5 during neurogenesis. However it was shown that 
row 5 it unaffected by Wg activity when the \textit{gooseberry} segment polarity gene, which is part of the Hh pathway, 
is expressed \cite{mcdonald1997establishing, Bhat1999}. 

The EGFr signaling pathway illustrates columnar patterning along the dorsoventral (DV) axis, playing a role in dividing 
the hemisegment into three columns that express different segmentation genes called the columnar genes, with EGFr signaling 
active in the medial and intermediate columns \cite{Skeath2003}. 
% each expressing a different columanr, namely \textit{vnd}, \textit{ind} and \textit{dr}. 

\subsubsection{Segmentation Genes}
Two key groups of segmentation genes were found to play a crucial role in the formation and specification of neuroblasts 
in the neuroectoderm, namely segment polarity genes on the AP axis and columnar genes on the DV axis. 
The specific combination of segmentation and columnar genes expressed in both the preneural clusters and the NBs themselves 
establishes the unique identity of each NB, thereby determining its fate \cite{Bhat1999, Skeath1999}. 
Segmentation genes of these two groups serve as markers to identify specific NBs, as different NBs express unique combinations 
of these gene, resulting in distinct expression profile that define each NB's identity \cite{doe1992molecular}. 

The activity of the segment polarity genes segments the neuroectoderm into four rows per hemisegment and can largely 
explain how NBs that develop in these different transverse rows acquire different fates \cite{Bhat1999, Skeath1999}.  
Segment polarity genes include signaling molecules (\textit{wingless (wg)}, \textit{hedgehog (Hh)}), 
transmembrane receptors (\textit{patched (ptc)}), transcription factors (\textit{gooseberry (gsb), engrailed (en))} 
that are part of the Wg and Hh signaling pathways \cite{Bhat1999}.  

Columnar gene activity was shown to subdivides the neuroectoderm into three longitudinal columns, 
ensuring that NBs that develop in different columns acquire different fates \cite{Skeath1999}. 
Columnar genes include the \textit{EGF receptor (EGFr)} mentioned above, and the three transcription 
factors \textit{ventral nerve cord defective (vnd)}, \textit{intermediate nerve cord defective (ind)}, 
and \textit{drop (Dr)} also known as \textit{msh}. \textit{Vnd}, \textit{ind} and \textit{Dr} are expressed in the medial, 
intermediate and lateral columns respectively \cite{mellerick1995dorsal,jimenez1995vnd,d1996msh,weiss1998dorsoventral, Skeath1999,Skeath2003}. 

\subsubsection{Neuroblasts Nomenclature}
As mentioned above, NBs within each hemisegments are initially organized into a grid of 4 rows and 3 columns. 
NBs continue to form in between this initial grid, and it becomes difficult to discern clear columns or rows of NBs, 
due to variability in NB positions \cite{doe1992molecular}. 
NBs are given numerical names based on their final position in the pattern. 
They get two numbers: the first one indicates the AP position (rows) and the second one indicates the mediolateral 
position on the dorsoventral axis (columns). For example NB 1-1 is the most medial NB of row 1. 
%Within each hemisegment at the beginning of the NBs formation  there are around 10 NBs organized in four rows and 
%three columns. By the end of the NBs formation there exist around seven rows of NBs \cite{Skeath1999, Crews2019}. 
The first appearing four NBs are numbered with odd numbers (1, 3, 5, 7) while the even numbers (2, 4, 6) are developed 
shortly later between them \cite{Skeath1999}.  


% TO DO: add Midline because appears in results!! MP2 -> Midline Precoursor 2 

% Drosophila Melanogaster has served as a powerful model organism for studying the intricate mechanisms underlying neurogenesis, 
% due to its well characterized genome, highly conserved developmental pathways (?), and fast development. 
% rapid generation time".  + good available tools. 
% Which signaling pathways are important for it and how. 

\section{Gene Regulatory Networks}
% add sentence about regulation processes
Gene regulatory networks (GRNs) are topological diagrams or graphs that model the regulatory interactions within a cell, 
where the nodes represent molecular entities such as genes, proteins, or RNA molecules, 
and the directed edges depict the regulatory relationships. 
Edges can also be signed to represent the nature of the regulation, for example positive or negative regulation 
\cite{Wilczynski2010, mercatelli2020gene}. 
Typically these networks include  the regulatory connections between transcription factors (TFs) and the genes they 
regulate known as target genes (TGs) \cite{Wilczynski2010}. 
TFs are DNA binding proteins that either activate or repress the transcription of their TGs when binding to their gene 
regulatory regions \cite{latchman1993transcription}. 
More specifically, TFs bind to regulatory regions of the DNA called cis-regulatory modules (CRMs), 
such as enhancers, silencer and promoters. 
CRMs control the activity of genes and usually contain numerous binding sites of different TFs 
\cite{levine2005gene, small1992regulation, davidson2001genomic}.  

Reconstruction of GRNs is crucial in systems biology for uncovering the regulatory mechanisms underlying 
biological processes. Essential cellular functions are driven by sophisticated interactions among genes, 
which regulate one another to ensure accurate gene product synthesis  \cite{chai2014review}. 
% This regulatory framework is a crucial aspect of the biological processes in organisms.

%\subsubsection{Gene Regulatory Networks of Developmental Processes}
Developmental processes in particular are guided by regulation of genes responsible for producing transcription 
factors and cell signalling pathway components.
The development of animal structures is governed by precise patterns of gene expression in the embryo, 
synchronized and location specific. Generally, complex patterns of gene expression require the interaction of 
multiple cis-regulatory modules \cite{levine2005gene, gray1994short}. Given that multiple TFs can regulate multiple modules, 
and each module may be affected by multiple TFs, gene expression regulation during development forms a complicated network 
\cite{levine2005gene}.

Computational tools are therefor essential to allow systemic analysis and understanding of complex and large scaled GRNs. 
This is particularly important given the increasing volume of data generated by advancements in experimental methods. 
These tools enable to investigate the behavior of GRNs under various conditions, assess their robustness, gain insight about 
their functionality in regards to their individual components and more \cite{karlebach2008modelling}. 

Modeling:  Mbodj et al. (2013)\cite{mbodj2013logical}

% Developmental processes in particular are guided by regulation of genes responsible for producing transcription factors and cell signalling pathway components.
% The development of animal structures is governed by precise patterns of gene expression in the embryo, synchronized and region specific. This complex regulation is achieved by activation and repression of genes in specific locations, primarily through TFs that bind to specific DNA regions known as cis-regulatory modules, such as enhancers and silencers, which control the activity of genes. Each cis-regulatory module typically consists of at least 300 base pairs and contains numerous TFs binding sites \cite{levine2005gene, small1992regulation, davidson2001genomic}.  Generally, complex patterns of gene expression require the interaction of multiple cis-regulatory modules \cite{levine2005gene, gray1994short}. Given that multiple TFs can regulate multiple modules, and each module may be affected by multiple TFs, gene expression regulation during development forms a complicated network \cite{levine2005gene}.

% spcifically GRN of cell differentiation

% \textbf{GRN in developmental biology - }Gene Regulation for Higher Cells: A Theory New facts regardingtheorganizationofthegenomeprovidecluestothenatureofgeneregulation.RoyJ.BrittenandEricH.Davidson
% ״The genomic program for development operates primarily by the regulated expression of genes encoding transcription factors andcomponents of cell signaling pathways. ״https://www.pnas.org/doi/epdf/10.1073/pnas.0408031102

% GRN in Drosophila embryonic development 
% Popular methods to model GRNs
% .
\subsection{Logical Modeling of Gene Regulatory Networks}
Several methods exist for modeling GRNs, notably logical and continuous modeling. Continuous modeling approaches, 
such as ordinary differential equations, use real-valued parameters over a continuous timescale to provide in-depth analyses 
of network behaviors. However, its use is limited to specific and usually smaller systems due to the need for extensive kinetic 
data and the computational constraints \cite{karlebach2008modelling, wang2012boolean}. In contrast, logical modeling, 
which is based on discrete functions, bypasses the requirement for kinetic parameters, allowing for qualitative, 
more global and computationally efficient descriptions of system behaviors \cite{wang2012boolean}.

Logical models were first introduced by Kauffman in 1969, who used discrete logic to model the biological process of gene 
regulation \cite{kauffman1969metabolic}.  It has become a key approach in systems biology, widely used for gene regulatory
 and signaling networks analysis. 
In multilevel logical modeling, the variables take a number of discrete states. However the most common form of this modeling 
uses Boolean networks, where variables represent the binary states of being inactive (0) or active (1). 
This discrete approach was shown to capture the essence of biological regulation, where interactions and regulatory 
effects typically take place only after the involved molecules reach a specific concentration threshold \cite{glass1972co}. 
Thus, this modeling approach mirrors the natural "on" or "off" states of genes based on their regulatory conditions, 
without delving into the specific examples of regulator and target molecule dynamics 
\cite{thomas1991regulatory,samaga2013modeling}.

% As a matter of fact, the logical framework proved useful in a wide range of biological applications: 
%cell differentiation in developmental processes (for instance, 
% drosophila development as in \href{https://www.frontiersin.org/journals/genetics/articles/10.3389/fgene.2016.00094/full\#B38}
%{González et al., 2008}; 
%\href{https://www.frontiersin.org/journals/genetics/articles/10.3389/fgene.2016.00094/full\#B90}{Sánchez et al., 2008}; 
%\href{https://www.frontiersin.org/journals/genetics/articles/10.3389/fgene.2016.00094/full\#B32}{Fauré et al., 2014}), 
%haematopoiesis (\href{https://www.frontiersin.org/journals/genetics/articles/10.3389/fgene.2016.00094/full\#B10}
%{Bonzanni et al., 2013}), T lymphocyte activation and differentiation (see Section 5.1), cell cycle control (see Section 5.2) 
%and more generally cell fate decisions such as proliferation, growth arrest, apoptosis, senescence, etc. 
%(see e.g., \href{https://www.frontiersin.org/journals/genetics/articles/10.3389/fgene.2016.00094/full\#B91}
%{Schlatter et al., 2009}; \href{https://www.frontiersin.org/journals/genetics/articles/10.3389/fgene.2016.00094/full\#B39}
%{Grieco et al., 2013}; \href{https://www.frontiersin.org/journals/genetics/articles/10.3389/fgene.2016.00094/full\#B67}
%{Mombach et al., 2014}; \href{https://www.frontiersin.org/journals/genetics/articles/10.3389/fgene.2016.00094/full\#B23}
%{Cohen et al., 2015}). ------> https://www.frontiersin.org/journals/genetics/articles/10.3389/fgene.2016.00094/full#B89

\subsubsection{Formalism of GRNs logical modeling}
%Add state of the art approaches of connecting sequencing data and specifically single cell data to mechanistic models!!!
Logical descriptions use variables with a discrete value, specifically 1 or 0 in the Boolean model case. 
The following formalism is based on the review done by Abou-Jaoudé et al. (2016) \cite{abou2016logical}. 
% Boolean model focus - because of the  mode at hand!

\subsubsection{Logical Model Definition}
A logical model $(G,K)$ of a regulatory network is defined as follows: 
\begin{itemize}
    \item   \textbf{$G = \{g_1, g_2, ...,g_n\}$:} A set of n  regulatory components (e.g. genes), 
        each $g_i$ is associated with an integer variable with a value in the range $\{0,...,max_i\}$, 
        defining the level of activity. In Boolean models $max_i=1$ for all $i$, hence variables are binary and equal either 
        $0$ or $1$, representing "OFF" or "ON" activity of the component.
    \item \textbf{State space $S=\prod_{i=1,...,n}\{0, ...,max_i\}$:} The combination of all possible states of $G$, 
        defined as the cartesian product of the ranges for each component $g_i$.  
        The model state is a vector $g = (g_1,...,g_n)$. The number of states in $S$ is always finite. 
    \item \textbf{Discrete transition function $K: S\rightarrow S$:} For each component $g_i$, a discrete function 
        $K_i: S\rightarrow \{0, ...,max_i\}$ define its value depending on the model state. 
        Therefor for a state vector $g$, $K(g) = (K_1(g),...,k_n(g))$. K therefor defines the behaviour of the model. 
        $K_i$ functions are logical functions, function that use the logical operators AND, OR and NOT. 
\end{itemize}

\subsubsection{Regulatory Graph}
The regulatory graph $(G, R)$ representing a family of logical models. It consists of nodes ($G$) representing the regulatory 
components, and signed directed edges  ($R$) representing activation or inhibition. The edges are defined by $K$, 
however it is worth mentioning that several logical rules sets can result in the same regulatory graph topology. 

\subsubsection{Logical Model Dynamics}
A logical model defines discrete dynamics over its state space $S$. Given a state $g$, the transition function $K$ 
specifies the possible changes of the model variables. 
% Model dynamics can be represented in State Transition Graphs (STG) where nodes are the state vectors and edges are 
% the transitions between the states.

\begin{itemize}
    \item \textbf{Initial state:} Represents the initial conditions of the system. 
    \item \textbf{Attractor:} Maximal set of mutually reachable states with no transitions leaving the set.  
    \begin{itemize}
        \item \textbf{Stable State:} When an attractor consists of only one state, $K(g)=g$, meaning the next state is 
            identical to the current state of the model. Each component value is maintained constant. 
            Stable states can often represent a phenotype, like a cell differentiated state for example.
        \item \textbf{Cyclic Attractor:} When an attractor consists of multiple states. May denote a biologically periodic 
            behaviour (e.g cell cycle).
    \end{itemize}
    \item \textbf{State Transition Graph}: A diagram of state vectors as nodes, and directed edges as transition to the 
        next state according to the logical rules.
It is usually practical to examine the whole state transition graph for smaller networks \cite{naldi2018logical}. 

\end{itemize}

\subsubsection{Synchronous vs Asynchronous Updating Schemes}
Many times a state $g$ has more than one variable to change its value.  
The order in which this happens is not trivial. The two most used updating schemes are \textit{synchronous} and 
\textit{asynchronous}. In synchronous update scheme all variables are updated simultaneously, hence there is an 
underlying assumption that all updates require the same time to take place, which is usually biologically unrealistic. 
Nonetheless it is widely used, due to its simplicity of application and interpretation, and low computational complexity. 
Alternatively, Thomas et al. (\cite{thomas1991regulatory}) introduced an asynchronous updating scheme, 
in which each variable is updated independently, yielding a transition per update variable. This results with non 
deterministic dynamics. Its advantage is it covers all possible transitions, including biologically meaningful ones. 
Its disadvantage is its more complex computationally, and harder to interpret, since not all states correspond to a 
biological meaning. 

\subsubsection{Related Work}
Early \textit{Drosophila} development has been the focus of a large number of dynamical modelling studies, 
however many times there are lacking kinetic parameters - "logical modeling of Drosophila signalling pathways" paper 
\cite{mbodj2013logical} .

\subsection{Gene Regulatory Network Inference from Single-Cell RNA-seq Data}
GRN inference is building a GRN network based on molecular data observations, under the assumption that the impacts of an 
actual underlying GRN are detectable and quantifiable within this data. \cite{badia2023gene}.
GRNs are traditionally constructed based on experimentally confirmed regulatory relationships, often found in literature or 
compiled in databases. Since validating every part of a GRN is challenging, many networks remain small, presenting a compromise 
between accuracy and completeness \cite{levine2005gene,Wilczynski2010}. Beyond experimental validations, GRNs have also been 
inferred from gene co-expression in bulk transcriptomics data. This approach offers less general and more context-specific 
insights when sufficient data is available, however transcriptomics data alone is not always enough for accurate gene 
regulation inference, and sometimes is better when combined with other data types, like chromatin accessibility. 
Moreover, bulk profiling does not distinguish between regulatory processes across different cell types within the same 
tissue \cite{badia2023gene}. 
The latter challenge can be addressed using since-cell sequencing (SC-seq) technologies. 
SC-seq enables the inference of GRNs considering different cell types \cite{klein2015droplet, macosko2015highly}. 
This advancement has led to the development of numerous new methods for inferring GRNs \cite{badia2023gene}. 

\subsubsection{Methods for inferring GRNs from Bulk Transcriptomics Data}
Methods inferring regulatory relationships from transcriptomics data generally aim to explain the observed variability 
in the expression of each gene by considering the expression of other genes. 

A popular and simple implementation of this approach is Weighted Gene Co-expression Network Analysis 
(WGCNA) \cite{langfelder2008wgcna}, which detects groups of genes that are expressed together by performing 
pairwise correlations across all genes in the transcriptome to form a co-expression network. 
However correlation alone produces an undirected network, which makes it hard to interpret as regulatory 
relationships between the genes and often results in many false positives \cite{badia2023gene}. 

GENIE3  \cite{huynh2010inferring}  and its faster implementation GRNBoost2 \cite{moerman2019grnboost2} are co-expression 
based methods  for inferring GRNs from gene expression data, that also incorporate prior knowledge from databases to 
distinguish TFs from target genes. This not only reduces the number of potential gene pairs to consider, but also yields 
directed results that are easier to interpret \cite{badia2023gene}. Moreover, both methods utilize ensemble-tree approaches, 
enabling them to capture complex regulatory relationships beyond linear correlations \cite{van2020scalable}. GENIE3 employs 
random forest models to predict the expression of each gene, using TF expressions as predictors. The models generate weights 
for the TFs, indicating their importance in regulating each target gene, with the highest weights suggesting regulatory 
links \cite{aibar2017scenic}. GRNBoost2 uses a gradient-boosting machines (GBM) algorithm instead of random forest, which, 
while similar in principle to GENIE3, offers the advantage of higher parallelism that significantly reduces runtime, 
making it more suitable for larger datasets \cite{moerman2019grnboost2}. 

Still, as summarized in the review of Badia-i-Mopel et al. (2023)  \cite{badia2023gene}, such unsupervised GRN inference 
methods based on transcriptomics data alone have only a moderate success in accurately inferring GRNs, primarily because 
they often overlook other regulatory mechanisms like chromatin accessibility, leading to numerous false positive results. 
Moreover, when these methods are based on bulk omics data, they are also not suitable for GRNs specifically for cell types 
or cell state. 

\subsubsection{Current Methods for inferring GRNs Single Cell Transcriptomics Data}
Single-cell technologies and single-cell RNA sequencing (scRNA-seq) in particular enable to examine individual cell 
types within tissues and hence allows a more detailed analysis of the GRNs that guide different cellular processes, 
like differentiation and specification \cite{nguyen2021comprehensive}. 
Suitable GRN reconstruction methods have been developed to infer cell type-specific TF–gene interactions \cite{badia2023gene}.

%SCENIC\cite{aibar2017scenic} is one of the earliest GRN inference methods designed specifically for scRNA-seq data. 
% It expands GENIE3/GRNBoost2 co-expression network approach. % In essence,It first uses GENIE3/GRNBoost2 to infer initial 
% co-expression modules. It then refines these modules into regulons of TFs and their directly regulated targets. 
% It does that by conducting TF-motif enrichment using RcisTarget \cite{...}, thereby defining groups of TF-target genes, 
% known as regulons. Finally, SCENIC assigns activity scores to these regulons in single cells, providing potential insights 
% into their specific regulatory functions in biological processes \cite{aibar2017scenic, badia2023gene}.
% exapnd about SCENIC method

Single-Cell Regulatory Network Inference and Clustering (SCENIC)\cite{aibar2017scenic} is one of the earliest GRN inference 
methods designed specifically for scRNA-seq data. SCENIC was implemented in R \cite{aibar2017scenic} and later also in 
Python \cite{van2020scalable}.% It expands GENIE3/GRNBoost2 co-expression network approach mentioned above. 
It consists of three main consecutive steps: Initial GRN inference, motif enrichment and cellular enrichment. 
% Two more useful methods are ... and ..... These method use pseudo time.
\begin{itemize}
    \item \textbf{Co-expression modules constructions:} Co-expression modules of TFs and candidate target genes are inferred 
        using GENIE3/GRNBoost mentioned above.
    \item \textbf{Motif enrichment:}  Co-expression modules are refined into regulons. In essence, \textit{i-cisTarget} 
        method \cite{herrmann2012cistarget} is used to predict which genes are more likely to be directly regulated by the 
        TF by using databases of genome-wide, cross-species motif rankings, to check if TF motifs are over-represented near 
        the transcription start sites of genes in the inferred module. 
    \item \textbf{Cellular enrichment:} \textit{AUCell} method \cite{aibar2017scenic} evaluates the enrichment of a regulon 
        in individual cells. It ranks each cell's genes based on expression levels, and measures for each regulons its relative 
        activity compared to the other genes within the cell. This score can be used in various ways, for example for 
        clustering the single cells, or for generating binary scores of activity with AUCell score threshold based on the 
        scores distribution across the whole dataset. 
\end{itemize}

Nguyen et al. (2020) \cite{nguyen2021comprehensive} conducted a survey in which they compared by simulation 15 available methods 
of GRN inference methods that use scRNA-seq data, including SCENIC. The comparison is based on three metrics: 
accuracy in reconstruction reference networks, sensitivity to different levels of dropout/sparsity of the given data, 
and time complexity. SCENIC to outperformed the other methods regarding the accuracy, and came in the top three in terms of 
time complexity and sensitivity. It was relatively effective in analyzing datasets with a high degree of sparsity compared to 
most other methods, but with relatively more variability in results as sparsity rises. 

% here add how good they did on some comparisons. SCENIC themselves reported their success, + nguyen2021comprehensive shwoed some good scores for them. Add something at the end to say why I chose scenic on the specific data. 

% For time stamped single cell data, pseudotime ordering characterizes  continuous changes and can be used to inform GRN inference. The resulting GRNs provide valuable insights into the complex processes involved in key fate decisions. LEAP\href{https://www.nature.com/articles/s41576-023-00618-5\#ref-CR53}{53} and SINCERITIES\href{https://www.nature.com/articles/s41576-023-00618-5\#ref-CR54}{54} are examples of GRN inference methods that leverage pseudotime ordering to infer the directionality between genes in the GRNs. The use of contrast-level statistics obtained after differential testing\href{https://www.nature.com/articles/s41576-023-00618-5\#ref-CR55}{55},\href{https://www.nature.com/articles/s41576-023-00618-5\#ref-CR56}{56} is an effective means of identifying differences between conditions, such as between healthy individuals and a cohort of patients with disease. 
% One cannot simply apply GRN methods developed for bulk sequencing to the analysis of single-cell data. This is due to several reasons. First, bulk analysis methods are typically designed to assess changes in bulk samples across conditions and thus cannot comprehensively assess differences between cell types in a spatial and temporal manner. Second, these methods are not efficient in coping with high levels of sparsity (dropouts) and the large number of cells in single-cell data. Therefore, a fast-growing number of GRN inference methods have been developed specifically for the analysis of scRNA-seq data \cite{nguyen2021comprehensive}.
% more specific to SCENIC: expands the methods developed for bulk transcriptomics to distinguish between different cell types. 
% add something about potential combinations of rna seq data / transcriptomics with other omics data for better more accurate results, like ataqseq and mention SCENIC+ as an example. 

%%% Combinaiton of GRN inference and boolean modeling
%%%%%% OPTIONAL: 
%"A Novel Boolean network inference strategy to model early hematopoiesis aging - Remy, 2022
% Used SCENIC as a way to infer the components of a Boolean network, and another tool to get the logical rules.